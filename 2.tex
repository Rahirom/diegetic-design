\subsubsection{Industrial Design}
 インダストリアルデザインという語は1920年代アメリカで使われだした工業製品のデザインを行うデザイン分野である。職業として決定づけたのはアメリカのレイモンド・ローウィであるが、ローウィのデザインは装飾的であり、現在のデザインの理解とは多少異なることに注意が必要である。

 起源はウィリアム・モリスが提唱したアーツアンドクラフツ運動に遡る。産業革命により大量生産が可能になる一方、市場には粗悪な製品が多く出回った。これに異を唱え、工芸品への回帰を主張した。

 アーツアンドクラフツ運動に影響を受け、1907年ミュンヘンにドイツ工作連盟が発足する。「芸術と職人技術の協力」を通じて工芸の「品位を高める」ことを目的とし「経済と芸術の統合」に取り組んだ。ドイツ工作連盟の背後には反歴史主義的市民啓蒙、輸出振興のためのデザイン改革の要請、民族主義の勃興など、当時のプロイセンにおける社会問題が存在していた。「質の高い仕事」による製品を求め、各分野の協調を奨励したが、これは今日的なデザインの根幹となっている。機械化・規格化を推進したヘルマン・ムテジウスと芸術的個性を求めたアンリ・ヴァンデ・ヴェルデが規格化論争で対立し、ヴェルデはドイツ工作連盟を去る。後年、ナチスによって解散させられる。

 ドイツ帝国が崩壊、ヴァイマル共和国が成立すると、1919年、工芸学校と美術学校が合併し、国立バウハウス・ヴァイマルが設立される。ムテジウスの元で働き、ヴェルテと親交の深かったグロピウスが校長に就任し、バウハウス創立宣言が出された。ヴァイマル初期には合理主義・機能主義と表現主義が混同した教育内容であったが、後に合理主義・機能主義がバウハウスの中心となる。これは今日までインダストリアルデザインの基本思想として受け継がれている。その後バウハウスはヴァイマルからデッサウ、ベルリンへと移っていったが、ドイツ工作連盟と同様にナチスの影響によって解散を余儀なくされた。その後、バウハウスの関連人物はそれぞれドイツ国外へ移動した。モホリ=ナジはシカゴにニュー・バウハウスを設立し、その教育方針はイリノイ工科大学へと引き継がれている。
 また、インダストリアルデザイン教育を行う日本の大学の多くはバウハウスの教育方針を参照しているため、日本におけるインダストリアルデザインはバウハウスの影響下にあるといえる。

\subsubsection{Design Thinking}
デザイン思考はデザインを科学の思考法として捉える。こうした捉え方は古くは認知心理学者のハーバート・サイモン著「システムの科学」、ロバート・マッキム著「視覚的思考の経験」などにみられる。ビジネスへの応用を試みたのはIDEOのデビット・ケリーである。問題解決プロセスとしてのデザイン思考は分析思考とは異なり、アイデアの積み上げによるものとされている。思考は7段階あるとされ、定義、研究、アイデア出し、プロトタイプ、選択、実行、学習である。これは同時発生的に行われることもある。原則として、人間性の規則、曖昧性の規則、再デザインの規則、触感性の規則が挙げられている。デザイン思考は集合知によって厄介な問題に対する解決を試みる。これらはチームによる共創を前提に提唱されているが、それ以前よりレイ・イームズなどの優れたデザイナーは自覚的であったかは不明ではあるものの、このプロセスを一人で達成していた。デザイン思考はデザイナーによって提唱されたものではなく、デザイナーが行うデザインプロセスを客観的に観察したものである。

 デザイン思考については提唱された初期と今日の理解において差があることには注意しなければならない。

\subsubsection{Interaction Design}
インタラクションデザインは技術的システム、生物学的システム、環境システム、組織などの振る舞いをデザインする。狭義にはコンピュータの画面遷移やGUIの振る舞いのデザインを指す。ユーザー調査を起点に、ユーザビリティや情緒的観点で評価される。広義に捉えた場合、コンピュータと人間だけに留まらず、人同士の対話や、人とそれ以外といった場面でも用いられる。認知心理学の基本原則がインタラクションデザインの基本原則となっているがヒューマンファクターやエルゴノミクスなどはインダストリアルデザイン、画面の設計はグラフィックデザイン、実装はコンピュータサイエンスと、様々な関連性が挙げられる。インタラクションデザインは広範囲に活用可能であるため、対象に応じて様々な領域の知識が必要となり、異分野提携チームで行われることが多い。そのため、デザイナーは関連する分野に理解がないと効率的に作業を行うことが出来ず、領域横断的なデザイン分野であると言える。

 ビル・モグリッジによって提唱された。1989年、ロイヤルカレッジオブアートにインタラクションデザインを扱うコンピュータリレーテッドデザインがジリアン・クラプトンスミスによって設立された。現在では、UIデザインやUXデザイン、サービスデザインなど更なる細分化が進んでいる。



\subsubsection{Critical Design}
 クリティカルデザインは現状を肯定しないデザインの態度である。アンソニー・ダンによって提唱された。方法論の定義がなく、実用的なデザイン手法とは言えない。デザインによる社会的影響とそのコンセプトを重要視する。これはリサーチから得られたデザイナー自身の考えを要求しているからである。クリティカルデザイン登場以前のデザインは現状を楽観的に捉えることが多い。これはデザイン思考においてもみられる兆候であるが、クリティカルデザインやスペキュラティブデザインは多くの問題は解決不能だと主張する。

 クリティカルデザインとアートの違いについて、アンソニー・ダンは、「クリティカルデザインは美術作品ではなく、ありうる可能性を考えるための試作品(ジェノタイプ)である」と主張している。


\subsubsection{Speculative Design}
スペキュラティブデザインは、アンソニー・ダンとフィオナ・レイビーによって提唱され、クリティカルデザインの発展上にあるデザインの態度である。前記の通り、多くの問題は解決不可能であるという態度の元、あり得るかもしれない未来を描き議論を促すことで、人々の考え方を変え、世界を変えることを基本思想としている。態度として提唱されたため、具体的な方法論はない。「思索的なデザインの提案は、既存のシステムの法律的・倫理的な限界を浮き彫りにする”探査”の役割を果たしうる」と主張する。

 デザインである必要性について、「デザインは、新しい技術的発展を、架空とはいえ信憑性のある日常の状況へと落とし込む力を持っている。」としている。

 多くの作品では、映像と「小道具」を組み合わせたものが制作物となる。「スペキュラティブデザインの小道具は、様々な理想、価値観、信念によって形作られる世界を、心の中に構築する”きっかけ”を生み出す」と主張している。

 ユーモアを重視すると主張するが、イギリスで提唱されたことには注意が必要である。

 アンソニー・ダンはSONYに在籍したことがあるプロダクトデザイナーであり、ロイヤルカレッジオブアートのコンピュータリレイテッドデザイン、後にデザインインタラクションズの教員であることからもスペキュラティブデザインはインタラクションデザインに起源をもつと考えられる。また、テクノロジーと人間の接点をデザインするという点においても同様である。

\subsubsection{Critical Making}
 クリティカルメイキングとはロードアイランド・スクール・オブ・デザインで実践されている教育。クリティカルシンキングとセットで活用される。クリティカルシンキングは、「これは情報を処理し評価すると同時に、過程に異議を唱え、様々な種類の知識を用いること」であり、領域横断的な思考を行うことで現状を批評的に考える。クリティカルメイキングは、これに加えて、身体知を活用し、モノに落とし込む。組み立て開始前に思考が完結している製造とは異なり、そのプロセス自体が拡張的で深い思考と発見を促す真剣な探求に対して、新たな可能性を開く、とされる。

\subsubsection{Conceputual Design}
 シド・ミードはロサンゼルスアートセンタースクール卒のインダストリアルデザイナーであり、シド・ミード社を設立して多くのクライアントを抱える一方、「スタートレック」、「ブレードランナー」等多くのSF作品のコンセプトアートを製作した。技術的な知識を持ってフィクション上のトランスポーテーションや街、メカトロニクスのデザインを行う。現実のプロダクト(フィリップス社など)をデザインする際もSF映画のデザインも、解決すべき問題は変わらず、プロダクトの周辺にあるストーリーが大切であるとインタビューで述べている。現在の多くのSF映像作品はシド・ミードの影響を受けていることからデザインフィクションもミードの影響を無視できない。

\subsubsection{Design Fiction}
 デザインフィクションは、ブルース・スターリングによって提唱された、「未来になっても何も変わらないだろうという考えを変えるための物語的プロトタイプ」である。未知のオブジェクトやサービスが生まれる可能性について考える機会を与えるために物語を利用する。SF映像作品という体裁を採ることが多いが、前述の目的を達成する為、フィクションのエンターテイメントではなく、デザインであると主張する。スターリングはスペキュラティブデザインとの違いについて、「スペキュラティブデザインは抽象的理想論に過ぎず、制作物を見せることよりもデザインの理念やプロセスを提示することに固執しているため、制作物を重要視するデザインフィクションとは異なる」と自身のブログに記している。

 一方で、スターリングはデザインフィクションは現状に対してクリティカルであることを強調しているため、クリティカルデザインの一部と解釈することも可能である。

 スターリングはメディアの進化を理解するためにデッドメディアの収集を試みた過去がある。これは著作、ディファレンスエンジンにみられるように、あり得たかもしれない未来を描くための思考法の一つである。

 デザインフィクションはデザインの分野でも研究され、マサチューセッツ工科大学メディアラボでは2013年からデザイン・フィクションズグループが組織された。スプツニ子!氏などが在籍したことで知られるが、2018年5月をもって解散した。

\subsubsection{Diegetic Prototype}
 デザイナーやエンジニア、リサーチャーからなるNear Future laboratoryのジュリアン・ブリーカーによって提唱された。ジュリアン・ブリーカーはデザインフィクションの実践者であり、その活動を広報する目的でブログを書いている。現在のコンテクストを参照して未来の状態を探り、予想外のもののチャートを作成し、新しい形のチャンスを与えることを掲げ、未来の状態を伝える方法としてDiegetic Prototypeを提唱した。これを受け、ブルース・スターリングはデザインフィクションの手法の定義としてDiegetic Prototypeという語を積極的に用いていることから、デザインフィクションの具体的な制作物がDiegetic Prototypeである。


\newpage
\subsection{過去の製作からの知見}

\subsubsection{”D.P.S.S.”(2015)}
 DPSSはデザイン思考によって制作されたQ-Drumに対してのクリティカルデザインとして考案した、砂を溶融させ成形することで水道管を出力しながら自立走行するドローンのデザイン案である。

 Q-Drumはアフリカの児童が水を運搬しなければならないことで教育を受けられず、その解決策として提案されたプロダクトである。ローラー状の容器によって一度に大量の水を運搬しやすくする。一方で、寄付に頼っていることで実装が難航している上、実質的に児童が水を運搬しなければならない。しばしばデザイン思考の例として取り上げられることがあるQ-Drumであるが、そのデザインが問題解決したとは言い難いと考えた。

 そこで、後発発展途上国が多く存在するアフリカ内陸国における実情をリサーチし、本作品のレンダリングイメージを制作し、背景とともにプレゼンテーションすることで、フィールドの実情やインフラ支援の難しさを伝えることを試みた。本制作を通して、実在する多くの厄介な問題を解決することは容易でなく、「デザインによる問題解決」は詭弁であるのではないかと考えた。

\subsubsection{洗濯の分散処理システム(2017)}

 本制作は洗濯機の方式が19世紀以降手作業から機械に置き換えられたものの、古代シュメール時代から変容がないことに着目し、洗濯のシステムの再考を試みたものである。洗濯の科学的な方式について考案することから発展し、都市型生活における洗濯の分散処理に至った。マルチサイドプラットフォームを多層化させることにより、ユーザー及び一般人のサービス提供者を含めた4つの職種による複層的なUXの検討を行った。洗濯を依頼するカスタマー、洗濯を実行するウォッシャー、都市内の運搬を担うキャリア、空間を貸し出すコンサベーターの4職種が複雑に絡み合うシステムとなったため、多様な働き方や都市生活を描画するためにビデオプロトタイプを用いた制作を行い、同時進行で複数の処理が行われる様子の描写を試みた。

\subsubsection{’25 convenience store(2017)}
 「2025年におけるコンビニエンスストアを提案する」という課題の元、制作を行った。テクノロジーの発展を盲信するのではなく、実体験に基づく架空の2025年像を構築し、そのフィールドの中で暮らす人々に必要になるであろうプロダクト及びサービスの提案をフィクションのストーリーテリングを通して行った。

 2016年、広島県因島地域を訪れた際、高齢者がコンビニエンスストアを交流の場として利用している様子を目撃した。当時、広島県因島地域は高齢化率が既に2025年の水準に到達していた。因島は離島であるため、先進的な遠隔医療の試験的な実験が行われていた。そこで、因島を2025年の日本の地方都市のモデルケースとして捉えて制作に当たった。

 2017年当時、在住していた大垣市ではコンビニエンスストアの閉店が相次ぎ、地方でのコンビニエンスストアの経営の難しさを感じていた。一方で、大垣市より小規模の集落などでは、コンビニエンスストアが唯一の小売店である地域も多く、自動車免許を返納した高齢者の生活のインフラとして作用しているのではないかと考えた。経営の難しさは24時間営業するための労働力の確保、売り上げの少なさなどが原因であると推定した。故に自動決済システムを導入した無人店舗とし、ワゴン型にすることで店舗自体を既存のコンビニエンスストアチェーンの流通システムによって輸送し、集落に届けることを軸とし、店舗のデザイン及び決済システムのプロトタイピングを行った。

 また、この実情を都会のデザイナーに伝えるために、評価するデザイナーの親世代に当たる限界集落に在住する後期高齢者の物語を構築した。プレゼンテーションそのものが自身の体験談を軸としたストーリーとして構成され、その中で架空の限界集落の高齢者の物語を伝えることで、二重構造のストーリーテリングを構成した。評価としては、ストーリーテリングやプロトタイプについては概ね好評を得られたが、実現可能性の低さを指摘された。

 尚、本作品のコンセプトデザインとほぼ同様のコンセプトを持つ「Moby Mart」が2018年スウェーデンで実用化された。

\subsubsection{Speculative Farmer}

 現状の問題が拡大した架空のフィールドを構築し、そこに暮らす人の生活をモキュメンタリーで撮影した実験作である。

 本作品では、2017年種苗法改正に伴い遺伝子組み換え作物の栽培が日本で普及した場合の農家について扱うこととした。このテーマの策定は問題の焦点となる遺伝子組み換え作物に対する日本人の認識が薄いこと、及び、具体的な確証や議論を経ることなく、遺伝子組み換え作物を否定する立場の世論が多いことに基づいて制作した。遺伝子組み換え作物の有効性及び危険性を双方説明した上で、製作者自身が岐阜県で米農家となった場合の未来像を構築した。米の遺伝子組み換えは現在一般には実現されていないが、南米諸国においては遺伝子組み換え小麦の栽培が活発に行われているとともに、育てた農家自身はそれを食べない現象が生じている。これを元に遺伝子組み換え作物の危険性を知りつつ、遺伝子組み換え米を栽培した上で、自身は有機農法の米を購入する農家を演じた。製作者は新潟県出身であり出身地で米農家を営む方が労力及び作付け面積あたりの単価が高く、岐阜県で米農家を営むことは現時点においては現実的でないが、遺伝子組み換え米が普及した場合、ブランド力のある新潟県においては非遺伝子組み換えの米が栽培され、岐阜県では手間のかからない遺伝子組み換え米の作付けが普及すると考えた。

 本制作からモキュメンタリーにおいては、虚構と現実の境界が曖昧になり、出演する人物のアイデンティティやバックグラウンドが意図に関わらず反映されてしまうことを確認した。更に未来像の細部の描画を追求することで、現在との相違、未来においても変化しないことについて伝えることが可能であると考えた。

\subsubsection{LACHESIS SYSTEM}
 本作品は、「何かを測定することで健康に貢献するシステムのインターフェースデザイン」という与えられたテーマの元に制作を行った。本研究においては手法の実験として本作品を扱うこととする。制作期間が3日と限られていたことから最低限度の制作を行い、後に再制作を行う。

 本作品のテーマに対し、「健康に貢献する」という文言が、健康であることを無条件に善とする思想があることに違和感を持った。それとともに、前提として示されている「測定すること」がそれに対しての方法としては安易なのではないかと疑問を持った。

 健康である状態とはどのような状態であるのか、という論点からアイデア出しを行った。自身を含むデザインを専攻する大学院生3人でブレインストーミングを行った結果、肉体と精神の健康については議論されたが、一般的に存在するとされる社会的な健康については議論が生じなかった。これは健康の社会的要因に対する我々の認知が不足していることに起因すると考えた。このことから健康の社会的要因の認知及び、改善を主たる問題として扱うこととした。WHO欧州の提示するSOLID FACTSを参照し、格差や居住地域、教育によって健康が左右されることを認識した。

 健康に貢献することを善とするのであれば、これらの要因を全て平等に解決する方法を考案する必要があると考えた。また、健康を保証することは日本国憲法にも含まれており、国家の根底ともなり得る事項である。故に本テーマに対しては社会的システムや思想、信仰までもがデザインの対象となり得ると考えられる。

 制作を始めるにあたり、アイデアスケッチを行い、社会的インパクトを横軸、実装可能性を縦軸に設定し各アイデアの評価を行った。その際、右上にプロットされたアイデアが、「画像解析により全ての行為を測定することで寿命を算出するシステム」を描いたスケッチとなった。実際には、一卵性双生児を追った調査により、生活習慣の寿命に対する影響度は約75\%、先天性の影響が約25\%であることが知られている。先天性の影響は遺伝子に刻まれており、生涯変化しない。この遺伝子を読み解き寿命を解析する技術は現在開発中である。近い将来において、このシステムが実現する可能性は十分に想定できる。

 これを軸に、このシステムが実現することで社会がどのような変化を迎えるかについてシーンスケッチを行った。この際作成したシーンスケッチをレム・コールハースの「S,M,L,XL」をモチーフにし、影響範囲ごとに分類した。最小の影響範囲であるユーザーの体験をS、非ユーザーの体験をM、経済や都市についてをL、最大の影響範囲である国家や倫理、信仰についてをXLとした。この分類により、原案であるアイデアスケッチ(S)から発展し、XLまで順を追ってシーンスケッチを行い、再びSへと再帰させることで、人々の生活の予測を構築した。XLから逆行する過程において、健康は義務になり得ると考えた結果、その立場に立って発送することにより、非ユーザー(M)が存在しなくなることについては再考の余地があるが、この方法論は、複雑な個々の事象を横断的に捉えることができ、繰り返すことで、増大し続ける日本の医療費、社会保障費、高齢化社会などを考慮しながら、倫理観まで踏み込んだ未来像を詳細に描くことができた。

 ここから実装された場合の製作者自身の未来の生活のワンシーンのモキュメンタリーの制作、生体情報を計測する機器のプロダクトデザイン、寿命を通知するアプリケーションのユーザーインターフェースのデザインを行った。

 制作を通して、寿命を知る行為は、自分自身の死を客観的に定義されることに他ならず、それを行うシステムの根幹が現在の技術で実装可能な統計学、社会疫学の特化型AIであったとしても、末端のユーザーにとっては神のような存在として見えるのではないかと考えた。また、神のような存在と認識されるかもしれないものであったとしても、それをデザインするデザイナーは、テクノロジーの発展と捉え、システムを理解しやすいものとしてデザインを行うと考えた。

 このシステム自体が「測定することによって健康に貢献する」ものであると同時に、人間の死に対する認識を揺るがすシステムとして存在しうる。また、ノーシーボ効果と呼ばれるが、人間は死に対する思い込みで死ぬことがある。寿命を宣言されることで宣言された通りに死ぬ可能性が存在するのである。つまり、このシステムは生活習慣の改善を促し、「健康に貢献する」とともに、死をもたらす可能性を有している。

 寿命を決定する役割を担う古代ギリシャ神話の女神である「測る者」ラケシスをシステムの名前とした。



ヴィクター・パパネックは、
\begin{quotation}
  課題解決の行動としてのデザインは、その定義からいって、唯一の正解を与えるものでないからである。つまり、解答はつねに限りなくあるだろう。あるものは<いっそう正しく>、あるものは<いっそう間違った>もの、といったふうに。
\end{quotation}
としている。
